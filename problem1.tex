\documentclass[12pt,letterpaper]{article}
\usepackage{fullpage}
\usepackage[top=2cm, bottom=4.5cm, left=2.5cm, right=2.5cm]{geometry}
\usepackage{amsmath,amsthm,amsfonts,amssymb,amscd}
\usepackage{lastpage}
\usepackage{enumerate}
\usepackage{fancyhdr}
\usepackage{mathrsfs}
\usepackage{xcolor}
\usepackage{graphicx}
\usepackage{listings}
\usepackage{hyperref}


\setlength{\parindent}{0.0in}
\setlength{\parskip}{0.05in}

\newcommand\course{SOEN 6011}
\newcommand\hwnumber{1}                  
\newcommand\NetIDb{ID-40092284}          

\pagestyle{fancyplain}
\headheight 35pt
\lhead{\NetIDa}
\lhead{\NetIDa\\\NetIDb}
\chead{\textbf{\Large Project F1: arccos(x) }}
\rhead{\course \\ \today}
\lfoot{}
\cfoot{}
\rfoot{\small\thepage}
\headsep 1.5em

\begin{document}

\section*{Problem 1. Project Description}


\begin{abstract}

   The arccosine of x is defined as the inverse function of cosine of x where x lies in the range of 
    $-1\leq x\leq 1\\$
    
    Domain and range: The domain of the arccosine function is from −1 to +1 inclusive and the range is from 0 to π radians inclusive (or from 0° to 180°).The arccosine function can be extended to the complex numbers, in which case the domain is all complex numbers. 
 
    \begin{figure}[!h]
    \centering
    \includegraphics[width=0.3\linewidth]{arccosx.JPG}
    \caption{Figure 1: y = arccos(x)}
    \end{figure}
    
    Properties of the function y=arccos(x) that make it unique from other inverse trignometric functions:
    
    \begin{enumerate}
        \item
    Domain is in the range of [-1,1].
     \item
    Range is part of  [0,$\pi$].
     \item
    It is neither even or odd function.
     \item
    It is a decreasing function.
    \end{enumerate}
    
\end{abstract}
\end{document}
 